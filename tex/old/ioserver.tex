%\documentclass[11pt]{article}
%\begin{document}
\section{\bf I\rm nput \bf O\rm utput(IO) Server}
\hrulefill
\\ 
The IOServer acts as the software's interface to the outside world. While it is true
that the software has been designed to run autonomously, a need was stated early in 
the project to allow the observation and alteration of virtually every aspect of the
software during pre and mid experiment. Physically, the communication between the
ground support hardware and the flight computer is achieved through a set of telemetry
links, one uplink and one downlink. The uplink operates at 1200 baud and the downlink
operates at 9600 baud. Requests and responses are transmitted in integrity validated
packets. Once the input packets are read in through a serial port into the IOServer,
they are deciphered and then the IOServer is responsible for querying and commanding
the other programs. There are also several tasks that the IOServer is solely responsible
for. These include providing a virtual shell that functions similar to a standard unix
shell, and providing streaming house-keeping information from the computer. The entirety 
of the IOServers responsibilities is implemented using four threads. Each of these threads
will be explained in the following section.
 
\subsection{Internal Structure}

The IOServer program is divided into four threads. These include the main thread, which
initializes the variables and threads and also responds to input packets, the shell output thread,
which listens for output from the virtual shell and sends it down the downlink as it arrives, the 
vtmudeck input thread, which listens for message from the vtmudeck indicating the arrival of different
timers, uplinks, and signals, and finally the streaming house-keeping thread which sends streaming
data about onboard voltages and temperatures down the downlink. Each of these threads will be discussed
in detail in the following sections.

\subsubsection{Main Thread}

The main thread contains the bulk of the IOServer code. This is because it not only handles the 
initialization of the variables and threads, but it is also responsible for reading in packets
and then responding the requests therein. This culminates in a large loop that contains an equally
large number of If statements. \\
\\
Initialization of the majority of the variables and threads should be self explanatory when looking
at the code. The important bits include the MIOPipe and PowerPipe as well as the VSHell objects. The
MIOPipe and PowerPipe do what one would expect, they provide a convenient interface for communicating
with both the MDAQ program and the PowerServer program. It is recommended that the source for these
classes be consulted to understand how they function. The VShell object is a virtual shell that allows
commands to be executed as if they were typed in a unix shell. This class will be described in detail later.
The important piece to understand now is the meaning of the parameter passed to the constructor, in this
case "/dev/SOUTPIPE". This string specifies the destination to which the shell's output will be sent. 
This program uses /dev/SOUTPIPE which is a fifo. By doing this, the fifo can be read later and the output
can be extracted.\\
\\
Next, the main thread enters the main loop and it waits to receive the first packet. Packets encapsulate
requests and responses and are read and written with Input and Output Packetstreams. Each of these classes
is described in detail later. The packets read by the flight computer can have one of four types. These 
types include SHELL which contains a command to be executed in the virtual shell, MDAQ\_RQS which is a request
for some action by the MDAQ program, PWR which is either a request to turn on or off a powersystem or is a
query to request the status of a powersystem, or finally it can be of type HK\_RQS which is a request for a
particular house-keeping value. Each of these types in turn contain a number of possible subtypes. These subtypes
are to numerous to list here, so it is recommended that the source as well as \bf H \rm ouse Keeping \bf L \rm ink
\bf P \rm rotocol(HLP) documentation be consulted to find out about specific options and their use. \\
\\
The main program also checks for the validity of the packets. It is possible that the packets could become corrupted
during transmission. Therefore, parity checking ensures that the integrity of a packet is known. Upon receipt of 
good packets, Good Acknowledgements are sent. Likewise, upon receipt of bad packets, Bad Acknowledgements are sent.\\
\\
When the main thread exits its main loop, it waits for the other threads to end before ending the program.

\subsubsection{Shell Output Thread}

The shell output thread is fairly simple. Just as the name implies, this thread waits for output from the 
virtual shell to become available and then sends it in shell packets down the downlink. In this program the
output from the shell is sent to /dev/SOUTPIPE, so this thread opens /dev/SOUTPIPE and continuously reads
it. Because packet data can have a maximum length of 255 bytes, it is possible for a shell command to 
produce more data than can fit into one packet. In this case, multiple shell output packets are sent
down the downlink.

\subsubsection{VTMUDeck Input Thread}

The VTMUDeck Input Thread listens for input from the VTMUDeck indicating the arrival of a Timer,
Uplink, or Shutter signal. Communication with the VTMUDeck is accomplished with an IOSPipe object.
This class is similar to the MIOPipe and PowerPipe classes, and as such, the source should be
consulted to see how these classes are used.\\
\\
After a message has been read indicating that a Timer, Uplink, or Shutter signal has been received,
the vtmudeck input thread's only required action is to format a message with the relevant information
and send it down the downlink to ground support.

\subsubsection{Streaming HouseKeeping Thread}

The streaming housekeeping thread uses the LMSensors Project's sensors library to read voltage
and temperature information from the motherboard. Using their library involves opening a config
file and using it to initialize their library. After this, a sensors\_chip\_name structure can
be passed to the sensors\_get\_feature() function to extract information from the appropriate
chip. On the Hercules EBX motherboard, this chip is the via686a. Relevant information includes
a 2.0V reading, 2.5V reading, 3.3V reading, 5V reading, 12V reading, and several temperature readings.\\
\\
These values are extracted once every two seconds and sent down the downlink.
\subsection{Helper Classes}

Several Helper classes are utilized to give the IOServer its full functionality. These classes include the
Packet class, InputPacketStream class, OutputPacketStream class, and the VShell class. Each of these classes
will be described in detail in the following sections.

\subsubsection{Packet}

The Packet class is used to encapsulate the transmission packet. Data, which is almost always in the form or 
requests or responses, is sent between the flight computer and ground support in discrete packets. Each packet 
contains a type field, subtype field, data length field, data field, timestamp field, and parity byte. These 
fields are capped
with start and stop characters, parity encoded, and sent as a byte stream across the up and down links. With the
Packet class, each of these processes is taken care of automatically. A packet can be created by passing it the
necessary information and then sent down an output stream using the sendPacket() method. On the other end of
the physical link, the packet can be read from an input stream using the readPacket() method. This method also
automatically calculates the packets parity byte and then checks this against the parity byte transmitted to 
check for packet integrity. It is recommended that the source be consulted to find the exact syntax of the
other accessor and mutator methods. It is also recommended that the HLP documentation be consulted to learn
about the specifics of the HLP Packet specification.

\subsubsection{InputPacketStream \& OutputPacketStream}

The InputPacketStream and OutputPacketStream classes are merely wrapper classes that are designed to 
increase the ease of using packets. After the object has been created using the correct stream
object, packets can be read and written by simply calling readPacket() or writePacket().

\subsubsection{\bf V\rm irtual(V) Shell}

The VShell class is used to provide access to a unix shell. In this case, the shell is actually a 
slightly restricted bash shell. The VShell class achieves this by creating a new bash process. This
process has its standard input, standard output, and standard error streams redirected in such a 
way that it can be controlled by this shell.\\
\\
A VShell object is created by passing a device or file where the output should be sent. This is usually a
pipe that can be read from a program. After the object has been created, commands are executing by passing
them to the execute() method. For example vshell.execute("ls") would list all of the files in the current
directory. When the VShell is finished being used, the shell should be closed by using the exit() method.\\
\\
As previously mentioned, this is a somewhat restricted bash shell. Although it should behave very similarly to
a regular bash shell, because it is one, the nature of the shell's creation dictates job control may be unavailable.
It is recommended that the software be thoroughly tested with a wide range of scenarios to determine what is 
possible and what isn't.


%\end{document}

