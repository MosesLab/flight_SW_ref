%\documentclass[11pt]{article}
%\begin{document}
\section{\bf M\rm ission \bf D\rm ata \bf A\rm c\bf Q\rm uisition(MDAQ) }
\hrulefill
\\ 
The Mission Data Acquisition or MDAQ program is probably the most essential
piece of software in the entire MOSES collection. It is responsible for 
orchestrating the part of the experiment that collects actual science data.
This involves multiple steps, first there must be a predefined exposure sequence
that dictates the number of exposures and the durations of each individual exposure
in the sequence. When an exposure sequence is initiated, the MDAQ program must operate
the shutter correctly, command the Read Out Electronics(ROE) to read out the images
on the CCD cameras, acquire the data with a sustained rate of 4MB/s, and finally it needs
to both write the data to disk in an acceptable format and send as much data as it can 
down the high-speed data telemetry link. Besides this main task, the program must also be
able to communicate with the IOServer and alter its behavior according to any such requests
received.
This process requires the combined efforts of 
many individual pieces. The rest of this section will focus on these pieces and how they
work together.

\subsection{Internal Structure}

The MDAQ program can logically be divided into 4 different sections. In general, these sections
correspond with the different threads, however one section is actually composed of five threads
with each doing virtually the same thing. These sections include the main thread, which initializes
all of the variables, starts the threads, and then waits for the threads to finish, the telemetry 
thread, which streams science down down the telemetry link as it becomes available after acquisition,
the interactive I/O thread, which allows interprocess communication between the MDAQ program and the
other pieces of software, and finally the fourth section is the data capture section, which executes
exposure sequences, acquires data, and write the data to the disk. Due to some clashes between system
signals and threads, each signal must have its own thread that waits for the signal to appear and then
starts the data acquisition process. Therefore, all five signal threads get lumped into one section
from a logical standpoint. An in-depth discussion on each of these sections will follow.

\subsubsection{Main Thread}

As previously mentioned, the main thread is mostly responsible for initializing variables, starting all
of the threads and then waiting for the threads to finish. This is no doubt one of the simplest sections,
but it still performs several critical tasks. To begin with, there are several attributes and settings 
that can be specified by a user before beginning the program. These settings are stored in a file, usually 
mdaq.conf, and they must be extracted and parsed at the beginning of the program. This process is facilitated
through the use of an InitAttr object. A discussion of the InitAttr class will be postponed until later in this
document. At this point it will suffice to understand that these attributes are retrieved at this point in the
code. Arguably the most important piece of information that is retrieved in this way is a set of sequence file names.
These file names define which sequence file is associated with any given signal. This is particularly useful,
because it
means that a user can have any number of possible sequences stored in separate files and can specify in the mdaq.conf file which sequences to use just before a run. 
\\\\
Next, Sequence objects are created by passing the sequence filenames as arguments 
to the Sequence constructor. It is useful to encapsulate sequence information in an
object because it makes performing operations on the sequence very convenient. They
can also keep track of the current position in a sequence while it is being executed.
All of these features are discussed in detail later. At the same time that the Sequence
objects are being created, they are also being associated with a signal number through
the use of an associative array. After this point, it becomes trivial to determine which
sequence to execute after receiving a signal.
\\\\
The Main thread also uses this initial setup time to ensure that the Read Out Electronics(ROE) is
ready to take images in a desired manner. It does so through use of a ReadOutElectronics object,
covered later. The ReadOutElectronics object is instructed to exit default mode, reset, and then 
exit default mode
again. The cause for this seeming redundancy is that commanding the ROE to exit default mode while in 
command mode has no effect. However, commanding the ROE to reset while it is in default mode causes a 
software error. Therefore, we exit default mode once to make sure that the ROE is commandable, tell it to reset
so that we know what configuration it is in, and then finally exit default mode again to make it commandable
in a known configuration.
\\\\
MDAQ then configures the program for the proper use of signals and starts all of the threads. At this point
the main thread has nothing more to do, so it waits for an exit signal. Upon receipt of this signal, the
other threads are instructed to finish, killed if necessary, and resources are destroyed and given back to the
system. This concludes the duties given to the main thread.

\subsubsection{Telemetry Thread}

Telemetry in MDAQ simply involves sending acquired data(images) out of the telemetry 
port as they become available. Unfortunately, this process isn't as easy as writing 
the data to disk. The reason behind this, is that the telemetry link operates at a 
much slower data rate than the storage disk or data acquisition hardware. In fact, it
is going to be impossible to send all of the data that we receive down the telemetry
link. Therefore, the telemetry process is merely concerned with the sustained transmission 
of as much data as possible.\\
\\
The driver design of the telemetry card makes writing to the telemetry port virtually as 
simple as opening and writing to a file. The telemetry port is specified as /dev/sfc0.
Accordingly this file/device is opened for writing at the beginning of this thread. The
only non file-like operation that is performed is an ioctl to the device that flushes the
transmission buffer. From this point on, the user should be able to treat this device as
a file, in theory.\\
\\
Filenames of files that need to be written to the telemetry port are obtained in the
telemetry thread from a queue of file names. These file names are written to the queue 
in a data acquisition thread after the files have been written to disk. In essence,
the telemetry thread is basically a big loop that waits for the queue to become not
empty and then systematically opens and writes the files to the port until the queue
becomes empty again. In the actual experiment, it is very likely that the queue will
remain empty during the beginning of the experiment, but remain populated once it 
begins to receive file names. When this loop is instructed to terminate by the main 
thread, it expunges the names of any untransmitted files from the queue.\\
\\
As a final note on the telemetry thread, it also formats the data in such a way that
on the other end, an observer can tell where one file ends and another begins. This 
is accomplished by adding a header and footer to every file. The header and footer
contain bitpatterns that are extremely unlikely if not impossible to appear in actual
data. The code should be consulted to learn the actual bitpatterns used.

\subsubsection{Interactive I/O Thread}

Communication with the MDAQ program from the outside world is accomplished through the
interactive I/O thread. This thread serves as an interface for querying and altering
the behavior of the MDAQ program while it is running. Commands to the MDAQ program
are acquired through the use of an MIOPipe object, which utilizes a system FIFO to
facilitate interprocess communication. It is recommended that other programs whishing
to communicate with the MDAQ program use the MIOPipe class to ensure convenience and
compatibility.\\
\\
At this point all of the different interactive I/O options will be listed, but it is
highly recommended that the MIOPipe class be used / consulted to understand how these
options are used:
\begin{list}{}{}
\item Set the sequence name associated with a signal
\item Set the output name used as a stub for file names
\item Get the Process ID for a signal / thread
\item Get the sequence name associated with a signal
\item Get the sequence information(exposure durations) for a signal
\item Get the output name used for stubs
\item Get the name of the currently executing sequence, if any
\item Get the length of the current exposure
\item Get the index of the current exposure in its exposure sequence
\item Get the status(on/off) of selftest mode
\item Get the status(on/off) of stims mode
\item Get the status(on/off) of telemetry output
\item Get the status(on/off) of producing channel 0 data
\item Get the status(on/off) of producing only positive channel data
\item Scale the current sequence by a specified ratio
\item Translate the current sequence by a specified delta value
\item Find and replace all occurrences of one exposure duration with another for the current sequence
\item Jump to a specified index in the current sequence
\item Save the current sequence configuration to a specified file
\item Find an exposure duration in the current sequence and jump to it
\item Begin / Resume the current sequence
\item Pause / Stop(if received twice) the current sequence
\item Stop the current exposure immediately.
\item Turn telemetry output on/off
\item Turn Channel 0 data production on/off
\item Turn Positive channel only data production on/off
\item Open the Shutter
\item Close the Shutter
\item Get the value of an ROE House Keeping parameter
\item Reset the ROE
\item Command the ROE to exit default mode
\item Command the ROE to enter selftest mode
\item Command the ROE to enter/exit stims mode
\item Get the Analogue Electronics parameters for the ROE
\item Set the Analogue Electronics parameters for the ROE
\end{list}
Again, to see how these options are used and implemented, it is recommended that
the MIOPipe and MDAQ source be consulted.

\subsubsection{Data Acquisition}
Each signal has its own thread which waits for the signal to arrive. Once the signal arrives,
the capture\_data function is called and this starts the data acquisition process. Inside the
capture\_data function, the process begins by loading the correct sequence. This is accomplished
by setting the current\_sequence pointer to point to the sequence associated with the received
signal number. Next, the function enters a for-loop that steps through the exposure sequence
using the Sequence.getNextExposure() method and performing Sequence.getExposureCount() iterations.\\
\\
Data Acquisition inside the for-loop involves several steps. First the shutter must be opened and closed,
remaining open for the desired exposure length. This is accomplished with the takeExposure() function.
The actual signals to the shutter are generated by a DIOBoard object, and the timing is achieved with
the gettimeofday() system call. \\
\\
Next, a DmaDaq object is told to expect data and the ReadOutElectronics
object is told to read out the data. At this point the ROE should be streaming data to the FPGA, which
will in turn be using Direct Memory Access(DMA) to place the data into main memory. Because DMA does not
require the use of the processor, the DmaDaq object is able to simultaneously sort the previous data as
new data is coming in. The data needs to be sorted because the data from all four channels is interlaced
as it comes down the stream. Thus it is necessary to place all of the channel zero data into its own
buffer and so on. After the call to DmaDaq.finishAndSort() has completed, the program should be left with
four buffers full of data. \\
\\
The final step is to write the acquired data to disk. This is easily accomplished with a call to write\_data().
Calling write\_data() requires the program to send in a pointer to an array of four buffers with data, a pointer
to an array of four ints representing the sizes, in pixels, of the four channels, and a Data\_Frame\_Info structure
that contains relevant information about the exposure and sequence. Inside the write\_data() function, an instance
of the RoeImage class is created to automate the both the formatting and writing of the data. This function also 
adds the new filename to the telemetry queue so that it can be transmitted if time allows. Both the DmaDaq and RoeImage
classes are described in future sections.\\
\\
This process occurs for every exposure. After the last exposure has been taken, the thread exits the capture\_data
function and once again begins listening for the appropriate signal to arrive.

\subsection{Helper Classes}

A number of non-standard helper classes are utilized to make the MDAQ program easier
to write, easier to understand, and easier to maintain. Each of these classes are
described in the following subsections.
  
\subsubsection{InitAttr}

The InitAttr class provides a convenient way to extract attribute information from
a file. Specifically, it extracts space delimited key-value pairs, where the key
always appears first. As the file is parsed in the constructor, the pairs are stored
in an associative array, so that the key can be easily used to fetch the value. \\
\\
If the type of data is for a key is already known, the InitAttr class also has methods
that automatically cast and convert the ascii value to the desired type. The following is
an example of what a file might look like followed by a typical program that might use the 
InitAttr class with this file.

\begin{verbatim}
**********************************
  #attr.conf

  val1 10
  val2 2.5

  printVals true

**********************************
 //initReader.cpp
 #include"initattr.h"
 ...
 InitAttr attr("attr.conf");

 int v1 = attr.getIntParam("val1");
 double v2 = attr.getDoubleParam("val2");

 if(attr.getBoolParam("printVals") == true)
     out << "v1 * v2 =" << (double(v1) * v2) <<endl;

**********************************
\end{verbatim} 
Output:
\begin{verbatim}
  v1 * v2 = 25.0

**********************************
\end{verbatim} 

\subsubsection{Sequence}

The Sequence class provides an object oriented representation of an exposure
sequence. Sequences can either be created by passing exposure information to
the constructor, or the sequence can be parsed from a file by passing a file
name to the constructor. Sequence files end in .asq, which stands for ascii
sequence. The structure of an asq file is relatively simple and an example
will appear at the end of this section. For now, it will suffice to know that
the file contains the name of the sequence, the exposure count, and a space delimited
list of exposure durations.\\
\\
A Sequence object can keep track of the current position in an exposure sequence
while the sequence is being executed. It also contains many methods for manipulating
the sequence in different ways. The following is a list of possible sequence operations:
\begin{list}{}{}
\item Get the sequence name
\item Get the file name
\item Get the exposure count
\item Get an exposure at a specified index
\item Get the current exposure
\item Get the next exposure
\item Find out if there are more exposures left
\item Get the current index
\item Find and replace one exposure duration with another
\item Scale all exposures by a specified ratio
\item Translate all exposures by a specified delta
\item Jump to a specified index
\item Find an exposure and jump to it
\item Save the sequence and any changes made to a specified file
\end{list}
Thus, the Sequence class makes working sequences very convenient. It is suggested
that the source for the sequence class be consulted to determine specific method
names and argument types. The following is an example of a sequence file:
\begin{verbatim}
  SEQUENCE:
          NAME: datadefault
          COUNT: 5
          BEGIN:
                  .5 1.0 2.0 1.0 .5
          END:
\end{verbatim}
\subsubsection{ReadOutElectronics}
The ReadOutElectronics class is meant to provide an easy and reliable way of 
communicating with the ROE. Upon creation, it attempts to establish a connection
with the ROE over a serial port that must be pointed to by /dev/roe. The ReadOutElectronics
class takes care of configuring the serial port to ensure correct communication.\\
\\
To begin using the ROE directly after it has been powered up, it is necessary to use the
ReadOutElectronics.exitDefault() method which takes the ROE out of its default mode, that being one
which is not commandable and reads out an image every eight seconds, and puts it into
command mode. \\
\\
In command mode, there are a number of different options that can be selected
to alter the way the ROE behaves. This may come as a surprise, considering that the ROE's
primary and virtually only job is to read out the data on the CCD cameras, but there are a number
of different ways that this data can be read out. For instance, if the ROE is placed in selftest mode
through the use of the selftestMode() method, the ROE will not read out the CCD's at all, but will
instead read out a predefined sequence of vertical bars. This can be useful for testing the flight
computers data acquisition abilities. Once the ROE is in selftest mode, it must be reset via the reset()
method in order to exit selftest mode. One would also need to exit default mode again after the reset.\\
\\
Another useful tool is the ROE's Stims mode. This is another diagnostic tool for checking the correct
functioning of the ROE. In this mode, a predefined pattern is generated in much the same way as the
selftest pattern. The difference is that in stims mode the pattern is fed through the CCD readout circuits
and thus any extraneous noise and other anomalies show up in the pattern. Also, unlike selftest mode, stims
mode can be exited by using the stimOff() method.\\
\\
Yet another tool that is provided by the ReadOutElectronics class, is the ability to easily query the ROE
for house-keeping data. The ROE has many voltages, currents, and temperatures that each have a unique H/K id
number. When this id number is passed to the getHK() method, the raw value for the H/K parameter is returned.
The types and id's for the ROE's H/K parameters are defined in the roehk.h file.\\
\\
This finally brings us to the bread and butter of the ROE, the flush() and readOut() methods. Flush is used to
clear any accumulated data on the cameras and readOut is used to get the data from the cameras. The biggest difference
between the two, is that flushing the ROE is faster and doesn't send any data down the data link. It is recommended that
the ROE be flushed five times in succession before every exposure. Readout does just what one would expect, it
reads out the cameras. It is suggested that the ROE be allowed four seconds for reading out data between exposures.
The only notable point of confusion for this method is that it also requires a block id, which is basically just
a byte that defines how the ROE needs to be read out. There are unique id's for reading out the ROE normally, reading out
while in selftest mode, reading out while in stims mode, and there are even some undocumented id's which were used for
testing during the construction of the ROE. All of these block id's are documented in the roe.h file.\\
\\
As a last point of interest, there are also methods for getting and setting the analogue electronics parameters inside
the ROE. These methods normally go unused and are usually only useful during testing. These parameters consist of 8
bytes which control the ROE's behavior. In fact, it is by setting these values that the ROE is placed into selftest
and stims mode. In those methods these parameters are written automatically. It is suggested that these parameters only
be changed if an experienced ROE operator / technician knows what they are doing.

\subsubsection{DmaDaq}
The DmaDaq class is mostly just an object oriented wrapper around the FPGA API defined in pcifio.h. As such, the
pcifio api provides in-depth coverage on most of the functions called in this class. The methods without procedural
equivalents include the constructor, the finishAndSort() method, and the flush() method. These methods will be
covered here.\\
\\
Most of the constructors activities involve initializing the pcifio board and fifo. After this initialization has
completed, the constructor goes on to setup proper DMA operation. When the computer is booted up, the boot loader
sets away a portion of memory towards the top for DMA. The constructor retrieves both a pointer to this memory, as
well as a logical address that is passed to the DMA functions. The constructor takes parameters specifying the 
buffer count and the buffer size. Basically, these values are used to partition the DMA buffer into several smaller
sections. These smaller sections are then used in a see-saw manner. While one buffer is being filled up with data
another is being readout. In the moses software, the number of buffers is set to 4. The size is set to be one fourth
of the data coming in one exposure. The reason for this is slightly complicated. In the pcifio api, there is a maximum
amount of data that can be read with one call to the card. This maximum value just happens to be equal to one fourth of
our data. Therefore, the function must be called four times. We use four buffers, so that every call to the card can 
have its own buffer and when the calls are finished the data lines up in one contiguous block.\\
\\
As previously mentioned, the read function must be called four times. Because the FPGA card needs to be listening for
data before the ROE starts reading out, one of these calls is made, the ROE is commanded to readout, and then the other
three calls must be made. These other three calls have been combined into the finishAndSort() method. The nice thing about
this method is that it also sorts the data while new data is being acquired. As previously mentioned in this document, the
data from the ROE comes from four different channels and is interleaved. The sorting that takes place in this method
untangles these channels and stores the data from each channel into and array of four pixel buffers which is passed
as a parameter. The method also updates and array of four index which is also passed as a parameter. When this method
finishes, these four indexes can be used to tell the sizes in pixels of the four data buffers.\\
\\
The last method that needs to be covered is also the simplest. The flush() method simply reads and discards any extraneous
data that is left in the physical fifo after previous read methods have read out the real data from the ROE. The 
information left in the fifo is often the byproduct of certain workarounds that had to used in the FPGA design.

\subsubsection{RoeImage}
Images from the ROE are written to disk through the use of the RoeImage class. At first glance this class might seem
unnecessary because the data from the ROE is saved in such a simple format. How Simple? Well to be honest, The first
channel is written if it is present, followed by the second channel, third, and fourth. This format was chosen because
software already existed for reading such a file into an image / data analysis software suite. The reason that we 
can't just write the data to disk and be done with it is that there is a lot of important information about an
image that shouldn't be lost. This is where the RoeImage class comes in. It implements an effective and efficient
way of storing this information.\\
\\
Here is how it works. After the data portion of the image has been set, a number of other methods are used to set
the relevant information. This information includes the name of the image( which is usually something like darktest1 ),
setting the number of bits per pixel, setting the width and the height, setting the exposure duration, setting the 
number of channels included, setting the date and time, and setting the origin, instrument, observer, and object for the
image. It now becomes obvious that this is a lot of useful information. To store this information it is important that the
image have a unique filename. This is easily achieved by using the date and time. The reason that the name needs to be 
unique is that the RoeImage class saves all of the image information in a common XML file. The file name is used as the
index into this file. This XML file can then be parsed by different software later or it can even be parsed by
the RoeImage class to read an image and its information from disk.

%\end{document}

