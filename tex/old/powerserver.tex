%\documentclass[11pt]{article}
%\begin{document}
\section{Power Server}
\hrulefill
\\ 
The PowerServer is responsible for turning on and off power to subsystems as well as keeping
track of the current power configuration. Changes and queries can be made during the experiment 
by communicating with the PowerServer through the PowerPipe class. This class uses a system FIFO
to send commands to the PowerServer. The PowerServer can also be used in conjunction with the 
startpower script and the powersys.conf configuration file to power on any necessary systems in
a predefined order at startup. These options will be covered later.

\subsection{Internal Structure}

Internally, the PowerServer is a continuous loop which reads command from a PowerPipe object and
uses a PowerSystem object to fulfill the commands. The PowerSystem class is merely a PowerServer
oriented wrapper around the DIOBoard class. So when you tell it to turn on a subsystem, you are
really just telling it to assert a DIO pin. The PowerSystem class uses DIO ports C and D.\\ 
\newline
There are four different types of commands. There are \it Up \rm commands that tell the PowerServer
to power up a system, \it Down \rm commands that tell the PowerServer to power down a system,
\it Query \rm commands which ask for the status of a system, and there is also an \it Exit \rm 
command. When the commands come out of the PowerPipe, they are in the form of a three character string.
The first character is either 'U' for up, 'D' for down, 'Q' for query, or 'E' for exit. The remaining 
characters are either "XT" in the case of the exit command, or they are numbers specifying the pin / system
that is of interest. For the \it Up \rm and \it Down \rm commands, the remaining characters can also be
"AL" specifying that all systems should be either powered up or down.

\subsection{Automatic Powerup at Startup}

As mentioned earlier, the startpower script can be used in conjunction with the powersys.conf configuration
file to automatically power up any desired systems at computer boot time. The script can either interactively
prompt the user for every system included in powersys.conf or it can be made to automatically start all systems
through the use of the -a flag. The powersys.conf file uses a very simple structure. Each subsystem must have a
name and number. The number is basically the system's DIO pin number. The name is only used for display purposes
in order to let the user know what is happening. The name and number must be separated by an equals sign and the
whole line needs to end with a colon. The systems are started in the order in which they appear in the file\\
\newline
As an example, Lets assume that we have device1, device2, and device3. They need to be started in that order and
they have been assigned DIO pins 5, 6, and 7 respectively. The powersys.conf file would look as follows:
\begin{verbatim}
  #powersys.conf

  device1=05:
  device2=06:
  device3=07:

\end{verbatim}
These devices will be powered up at startup if the 'startpower -a' script is run after the PowerServer
has been started.

%\end{document}

