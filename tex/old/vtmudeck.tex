%\documentclass[11pt]{article}
%\begin{document}
\section{\bf V\rm irtual \bf T\rm i\bf M\rm er and \bf U\rm plink(VTMU) Deck }
\hrulefill
\\ 
The VTMUDeck is the interface through which timers and uplinks gain entry into 
the system. In essence, this program is nothing more than a loop which poles 
several banks of digital I/O pins looking for state changes and taking the appropriate 
actions when these changes occur. Currently, there are five predefined timers, 
nine uplinks, and one somewhat unrelated shutter signal. These signals are read 
in on DIO ports A \& B using the DIOBoard class. Changes in the pins are exposed 
by taking the exclusive or of the new pins and the old pins. Any bits that have 
changed become ones while those that have stayed the same become zeros. Currently, 
the software only supports having a maximum of one bit change on any given iteration. 
Fortunately, due to the nature of the Timers and Uplinks, having concurrent signals 
is very unlikely, and would most likely be the result of a very unusual situation. Also, 
the software has roughly one millisecond timing resolution, which should be sufficient 
for its target environment.
\\
\\
Pin position must be calculated to discern the meaning of the input. As the input is
 read in as an unsigned integral 8 bit value, the position of the changed bit can be
 calculated by the $log_2$ of the value. The initial result which is of type double 
must be converted to an int. Due to the internal representation of doubles, it is possible 
for a value such as 15 to be stored as 14.99999... This poses a problem because a conversion 
to int will merely truncate anything past the decimal point. In the previous example, 
$log_2$ 128 becomes 14.99999 which is converted to 14! To avoid this, we take the value 
multiplied by 1000 plus one, then we divide by 1000 to get the final result. So, 14.99999 
becomes 14999.99 = 14999 + 1 = 15000 / 1000 = 15. This accomplished by the following code:

\begin{verbatim}
        temp = (board.readPins(DIOBoard::PORTA) << 8 | 
                board.readPins(DIOBoard::PORTB) );
        diff = temp ^ pins; 
        double dindex = log10(double(diff)) / log10(2.0);
        int index = (int(dindex*1000)+1) / 1000;
\end{verbatim}
Where pins was the value of pins after the previous read.

\subsection{Timers}
As previously mentioned, there are five predefined timer signals. These include \it Data 
Start, Data Stop, Dark 2, Dark 4, \rm and \it Sleep\rm . All but the \it Sleep \rm timer 
are used to initiate exposure sequences. The difference between the Dark Sequences and the 
Data Sequence is that only the Data Sequence opens the shutter during an exposure. It 
should also be noted that the \it Data Stop \rm timer serves as a dual purpose timer. It not only 
instructs the Data Sequence to stop, but it also initiates the Dark 3 Sequence. When any 
of these signals are detected, the VTMUDeck uses MIOPipe to convey the signal to the MDAQ 
Program. Also, the IOSPipe is used to inform the IOServer of the signals arrival.
\\
\\
The \it Sleep \rm timer is used to put the computer to sleep while it is coming back down 
to Earth. This is mostly concerned with conserving power so that there is enough left to 
extract the data once the payload is recovered. The sleep command first uses the IOSPipe 
to inform the IOServer of the signals arrival. Next it forks to create a new process. This 
process is then overwritten with the 'apmsleep' command by using the execlp system call. 
When given the '--standby' parameter, the 'apmsleep' command proceeds to put the computer 
into standby mode until it either receives an interrupt or exceeds its timeout limit. This 
limit is given as a parameter in the form "+hh:mm". In this program the limit is set as 20 
minutes. The command as a whole appears as follows:
\begin{verbatim}
  if(fork() == 0)
    execlp("apmsleep","apmsleep","--standby","+00:20",(char *)0);
\end{verbatim}

\subsection{Uplinks}
There are currently nine uplinks that the VTMUDeck is programmed to respond to. Many of these
only serve as an added layer of redundancy and perform virtually the same task as one of the
predefined timers. The benefit to having the uplinks is that they can be issued at any time and 
can be issued multiple times, should an earlier uplink / timer fail or need to be repeated. Besides
the \it Data Start, Data Stop, Dark 2, Dark 4, \rm and \it Sleep \rm uplinks, which perform obvious
functions, there are \it Dark 1, Dark 3, Wake, \rm and \it Test \rm uplinks as well. There might 
be some curiosity at this point as to why \it Dark 1 \rm does not have its own timer. The reason
is simple. As it is currently planned, the Dark 1 Sequence will be taken while the payload is still
on the rail. Therefore, there is no need, and in fact it would make no sense, to have a pretimed timer.
The \it Dark 3\rm timer should be self explanatory.
\\
\\
This brings us to \it Wake \rm and \it Test \rm. To be honest \it Wake \rm does not wake the system as you might
naturally assume. In order to execute the \it Wake \rm code, the computer must allready be awake. Therefore,
the \it Wake \rm code only serves to inform the IOServer that the computer has been awoken. In some sense the 
\it Test \rm uplink is very similar to \it Wake \rm. It too only serves to inform the IOServer of its arrival. This
might make more sense for \it Test \rm though, because its purpose is to test the responsiveness of the computer and
in particular the IOServer.

\subsection{Shutter}

Finally we come to the shutter signal. This signal indicates whether the shutter is open or closed. In this sense,
it is the only pin that we would also like to know when it goes off as well as when it comes on. Therefore, there are two
nearly identicle pieces of code, one for when the pin goes high and one for when it goes low. In both cases a highly 
precise timing is acquired after this signal has been received. This timing is then sent to the IOServer along with 
a specification 
as to whether this is a shutter open or close signal, and all of this information is formated and sent down the H/K Downlink.
It can be used later to determine the acuracy of our exposure durations.
\\
\\
It might be helpfull to note that high resolution timing is accomplished through the unix gettimeofday() function. This
function is particularily usefull because it can achieve microsecond resolution. It is the basis for all high resolution
timing in this software.

%\end{document}

