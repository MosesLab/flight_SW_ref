%tex document
%Author: Reginald Mead
%Moses Flight Computer

\documentclass[11pt,titlepage]{article}
\title{MOSES Flight Computer}
\author{Reginald Mead}
\date{24 April 2005}

%Maybe these should be removed
\oddsidemargin 0.0in
\evensidemargin 0.0in
\textwidth 6.5in
\textheight 9.0in
%\hoffset -0.25in
\marginparwidth 0.0in
\marginparsep 0.0in



\begin{document}
\maketitle

\pagenumbering{roman}

\begin{abstract}
The Moses Flight Computer is a combination of hardware and software designed to operate the Multi Order Solar EUV Spectrograph being designed at
Montana State University by Dr. Charles Kankelborg. This instrument will be flown as a rocket payload in July 2005 and as such, the Flight Computer
must be designed and implemented accordingly. Designing and implementing hardware and software to meet the many requirements of this system is the 
focus of this paper. The Flight Computer appears stable at this time and requirements fulfillment seems probable. This project is ongoing and much 
progress has been made. It is currently on schedule for launch in July. The results so far are covered at the end of this paper and are accompanied 
by an analysis when possible. A short discussion of the remaining tasks in this project concludes this paper.

\end{abstract}

\tableofcontents
\newpage



\section{Introduction}
\pagenumbering{arabic}

In the study of the sun, the Multi Order Solar EUV Spectrograph promises to bring further insight into the inner workings of our local solar furnace. This
complex experiment is tentatively scheduled to launch as part of a NASA rocket payload in July 2005. The operation of the experiment requires an intricate
mesh of digital communication between subsystems. Therefore, a well designed, implemented, and tested Flight Computer is required to orchestrate the 
correct operation of this instrument. The experiment will only have five minutes to take data, so it is imperative that the Flight Computer work flawlessly.
The design and implementation of this system are the focus of the rest of this paper.


\section{Background}

\subsection{Multi Order Solar EUV Spectrograph}

The Multi Order Solar EUV Spectrograph(MOSES) is an exciting new instrument being designed at
Montana State University, under the direction of Dr. Charles Kankelborg. MOSES is a complex
project that hopes to give solar physicists a new and powerful tool with which to study the sun.

\subsubsection{Mission}

Spectrographs are instruments that break light into its component wavelengths and record this information in some way. 
This spectral information can be very useful because different wavelengths of light are produced by different elements or element ions.
By observing the spectral information of the sun, we can get a glimpse into its inner workings. Even more information can be extracted by
examining characteristics of the different wavelengths. Subtle differences can reveal information about the light source including velocity,
temperature and density, just to name a few. 


Spectrographs vary in design from model to model. One of the most common types of spectrograph is the slit spectrograph. In this model, a slit 
of light enters the spectrograph and diffracts into its component wavelengths. The received image can be graphed with a y-axis and a lambda-axis.
This type of spectrograph is probably the easiest to build, but it has an obvious flaw; an image can only be recorded one slit at a time. This 
makes capturing a full image very time consuming. In the case of the sun, the instrument must actually be swept across the disk of the sun. Further
complications are introduced by the inherently dynamic nature of the sun. The sun is naturally turbulent, much of which adds to its mystique. Physicists
naturally desire a full spectral image of the sun with data obtained from a single moment in time. This is impossible with a slit spectrograph. The 
nature of the instrument causes one side of the image to contain data from an early point in time than data found at the opposite side of the image.
The current solution comes in the form of the slitless spectrograph.

The slitless spectrograph works in much the same way as the slit-spectrograph. The obvious difference being that the slitless spectrograph has no slit 
and as such allows an entire image of the sun to enter the spectrograph. This image is then routed through a diffraction grating or reflected off of a 
special diffracting mirror. An entire image appears for every component wavelength of light and they appear at different diffraction orders. The separate 
orders correspond to varying amounts of diffraction. The zero order appears in the middle of the detector and has each wavelength exactly overlapping the
others with zero diffraction. Next out are the +1 and -1 orders which appear on opposite sides of the zero order. These orders show diffraction between
the component wavelengths with the +1 order mirroring the -1 order. In other words, a wavelength that diffracts to the right of another wavelength in the 
-1 order will diffract to the left of the corresponding wavelength in the +1 order. This continues with the +2 and -2 orders and so on.

Slitless spectrographs have their own obvious flaw. Complete images of the sun in different wavelengths overlap each other. Furthermore, the extraction
of data from this conglomerate image can be extremely difficult. This is one of the biggest reasons that slit-spectrographs are still so widely used. 
However, MOSES attempts to solve many of these problems and produce a complete co-temporal image of the sun.
 
\subsubsection{Procedure}

MOSES is a slitless spectrograph and shares many similarities with the slitless spectrographs that have come before it. However, there are several 
differences that make the MOSES instrument unique. The first difference is that MOSES is a multi-order slitless spectrograph that actually images more 
than one diffraction order. Previous multi-order slitless spectrographs usually only image one diffraction order, typically the +1 order. MOSES images
the +1, 0, and -1 diffraction orders on three separate Charge Coupled Device(CCD) cameras. Another important difference is that MOSES allows only a 
very narrow band of light
in the extreme ultra-violet spectrum(EUV) to enter the instrument. The only two prominent wavelengths are the SiXI and HeII wavelengths. This means 
that there should only be two overlapping images on any of the CCD's. The importance of these two characteristics should not be underestimated. 
Through the use of a new data extraction process developed by Dr. Kankelborg, the information from these three diffraction orders can be separated 
into their component images.

The MOSES experiment is tentatively scheduled for July of 2005. Because MOSES operates in the EUV spectrum, it must be flown on
a rocket that will take it above the majority of the EUV devouring atmosphere. Once it reaches its target altitude it will take a series of images
over the span of 5 minutes before returning to earth.


\subsubsection{Hardware Overview}

MOSES is composed of several subsystems, which each consist in themselves of a variety of different components. Several of the most important
components include the shutter, filters, mirrors, CCD cameras, read out electronics, and flight computer. These components form the bulk of the instrument 
and through them the data pipeline can be followed from input to output.

Data initially enters the instrument in the form of light. The shutter opens and closes allowing a certain amount of light to enter the system.
Thin film filters block the direct sunlight and ensure that only the HeII and SiXI 
wavelengths are allowed to continue through the instrument. This filtered light is then diffracted off of both a primary and secondary mirror which 
have both been specially manufactured to an extremely high degree of quality and have also been coated in order to reflect the EUV light. The mirrors 
direct the diffracted images onto three CCD cameras. This is the end of the data's journey in light form.

The CCD cameras convert the light into electrical charges which can be read out as a 2048x1024 grid of electrical signals. Each grid element accumulates
an electric charge that is proportional to the amount of light that hits the element. Next, the Read Out Electronics(ROE), a piece of equipment developed
at Mullard Space Science Lab, reads out the values from the CCDs and converts them into a serial stream of digital values.

Finally, the data is sent through a serial to parallel converter and read in by the Flight Computer where it is both stored for later retrieval and 
transmitted via telemetry back to a ground station on earth. Besides dealing with data, the Flight Computer is also required to control many other
aspects of the project. Designing and implementing a system to fulfill these requirements will be the subject of the rest of this paper. 


\subsection{Flight Computer Hardware Requirements}

The Flight Computer's functionality is facilitated by a combination of hardware and software, with the hardware forming the foundation upon which
the software will be written. The payload of a rocket is a very constrained environment and as such, there were certain requirements that hardware 
had to meet before it could even be considered. Beyond the base requirements, requirements were added to facilitate the basic operation of the instrument.
Requirements were also added as the result of tasks being moved from dedicated hardware to the Flight Computer. These decisions were often made based on
the cost and time required to manufacture the dedicated hardware. The following subsections describe each of the requirements in turn.

\subsubsection{Compact and Lightweight}

One of the constraints that naturally goes along with a rocket payload, is a fairly firm budget for mass and space. As far as
mass goes, there is a firm limit set by NASA on how heavy the payload can be. It is important to meet this requirement from the fundamental
standpoint of meeting the launch criteria. There is also motivation for this project to not only meet this requirement, but to exceed it. The
benefit being that the lighter the payload, the higher the rocket will reach. For our project, higher altitude means longer and better observing
time, both of which are naturally desirable.

In terms of space, almost every piece of the payload is mounted to either side of an optical table donated by Lockheed Martin. With much of the
space on one side of the table remaining open and unobstructed to facilitate the experiment, the remaining table real estate is a precious commodity.

\subsubsection{High-speed Low Power CPU}

High processing speeds and low power consumption are desirable in any computer system. Unfortunately, the general rule is that the two are
inversely proportionate. If only one of the two was required the other could be forfeited to reach optimal results. However, both are required
for this project, which means that a balance must be reached. The processor needs to powerful enough to process the large amounts of data 
but it cannot be so powerful that it exceeds the system power budget. As if this isn't bad enough, another limitation also enters into the mix.
Because much of the power consumed by the system is converted to heat, there is a thermal issue that must also be dealt with. When the payload
is flown, it will be in vacuum. This is a problem for most processors because they rely on air to cool themselves and they break if they
get too hot. The solution is finding a processor that runs fairly cool. The heat that is produced can then be drawn away from the processor
with metal braids and dumped into the computer case or some other heat receiving object.

\subsubsection{High-speed Data Acquisition}
MOSES produces a lot of data. Consider that each CCD camera provides 2048x1024 16 bit pixels and the ROE is actually designed to readout four
CCDs. The data from each CCD is interlaced on its way to the computer, which means that the unused channel cannot simply be ignored. The 
result is that 16MB of data needs to be acquired by the computer over the course of a four second readout period. This equates to a data 
acquisition rate of 12Mbps at the computer. This is a fairly high rate and appropriate hardware is required to facilitate the acquisition.

\subsubsection{Solid State Data Storage}
Data and software need to be stored before, during, and after the experiment. This storage must be solid state for a very important reason.
As previously mentioned, the payload will be in vacuum during the actual experiment. This causes a serious problem for normal hard drives. 
Normal magnetic disk hard drives are designed around the assumption that the device will be operated in air. Therefore, the magnetic head
that reads data from the magnetic platters is designed to ride on a pillow of air while the platters are being rotated to the right position.
When air is removed from the system, the magnetic head buries itself in the platter and causes irreversible damage to the device. For this
reason a solid state device is required for this project.
 
The capacity of the device is also important. The amount of data to be collected is estimated at about 750MB. Because all of this data needs
to be stored on this device, it must be able to accommodate the 750MB of data as well as all of the software to run the system.

\subsubsection{High-speed Serial Output}

A certain degree of redundancy is good for most scientific applications. One redundancy in MOSES is designed to increase the probability
of recovering collected data. Even though it is very probable that the payload will be recovered in-tact and the data will be preserved on the
solid state storage, there is always the chance that something goes wrong and the payload is destroyed. Therefore, a design criteria was made that
at least some of the data should be transmitted via telemetry back to the ground station. NASA has provided the MOSES project with a telemetry 
channel capable of transmitting at 10Mbps for this purpose. Therefore, the flight computer needs to have some way of outputting the data in serial
format at 10Mbps.

Some readers may have noticed that the data is being acquired at 12Mbps, but is only being transmitted at 10Mbps. The obvious question arises of 
whether all of the data will be transmitted. The answer is no. The telemetry is a secondary system only intended to guarantee that some data will be
retrieved, even if the solid state storage fails or is destroyed.

\subsubsection{Digital Signal Processing \& Generation}

Moses relies on digital communication between its internal components. Many of these components need to communicate with the Flight Computer and in turn
the Flight Computer needs to communicate with many components. Therefore, it is essential that the Flight Computer have some way of performing digital
input and output. Specifically, input from the Timer \& Uplink deck is required to ensure that the computer performs different operations at the correct
times and output is required to control both the power subsystem and the shutter.

\subsubsection{Multiple Low Speed Serial Transmission Channels}

Several low speed serial ports are required for advanced communication within MOSES. All of the serial ports need to conform to the RS422 specification
for differential serial communication. One of the lines is required for commanding the ROE. Some of these commands are simple one or two byte commands that
simply tell the ROE to do something like flush the CCDs or to reset. Other commands are more complex and can configure the ROE in a number of ways, many
of which are used for trouble shooting and diagnostics. This line also needs to be capable of sending at 9600 Baud.

Another set of serial lines are needed for in flight communications with the ground station. These lines connect to another telemetry deck that will transmit
the signals back to Earth. One line is strictly an uplink while the other is strictly a downlink.
These communications lines will be used for any manual overrides, should such actions be necessary, as well as querying the system for useful information
and for streaming housekeeping information. Housekeeping refers to information having to do with the overall status of the system. Both of these lines must
also be RS422. The downlink needs to be capable of operation at 9600 Baud, but the uplink only needs to operate at 1200 Baud.

\subsubsection{Efficient DC-DC Power Supply}

All of the power within MOSES is supplied by a massive 28V battery stack. Computers require several different voltages to operate, none of which are normally
28V. Therefore, the Flight Computer needs to have a DC-DC power supply that is capable of taking either the 28V straight from the battery or 12V from a 
separate DC-DC power supply within MOSES and converting it to the voltages required by the Flight Computer. Again, because there is a strict power budget,
the power supply should be as efficient as possible.

\subsubsection{Onboard System Hardware Housekeeping}

The Flight Computer also needs to be able to monitoring and measuring important system housekeeping information such as voltages and temperatures. These
values will be monitored during testing and during the experiment to ensure that the system is working correctly. Unstable voltages and temperatures
within the system can cause hardware to fail or permanently break.

\subsubsection{Umbilical Support}

Finally, some form of umbilical port is required for making a physical connection with the payload after it lands for the purpose of extracting the data.
A connection between this port and the outside of the rocket will allow for the fastest data extraction. This is important because the systems power will
only last a short time after landing. If the data is not extracted before power failure, the task of retrieving the data will become a lot more laborious.


\subsection{Flight Computer Software Requirements}

Hardware is of little use without software to control it. The software is the part of the Flight Computer that orchestrates the entire experiment. 
Many of the software requirements are directly related to hardware requirements. Other software
requirements are designed to ensure that the software can work together.

\subsubsection{Operating System}
The operating system is the intermediary between the user software and the hardware. Finding the right operating system for this project is imperative.
There are several requirements for an operating system being used on MOSES. First, it must be compact. Most solid state storage has far less capacity
than an equivalent device with moving parts. The operating system must leave enough space for the data. Second, the operating system must be adaptable.
It is important that we can tailor the O.S. to fit our needs. Third, the operating system must be stable. Crashing in the middle of flight would be 
disastrous.

\subsubsection{High Resolution Timing}

Much of the value of the data that is collected is dependent upon knowing exactly how much light was allowed to enter the instrument. The amount of
light that enters the system during each exposure is controlled by the shutter which in turn is controlled by the Flight Computer. Therefore, the 
computer needs to be able to be able to control the shutter with a high degree of accuracy and more importantly it needs to be able to determine the
length of time that the shutter remains open. The mission specification requires the shutter to be controllable to within 10ms with the accuracy known
to within 1ms. To achieve this, some form of high resolution timing is required.


\subsubsection{Data Processing \& Storage}
 
Software is required to control the shutter and to perform the data acquisition as well as sort and filter the data. There must also be software 
written to automate the storage
and transmission of the data. This software will have to interface to both the data acquisition hardware as well as the high speed serial output 
hardware. Sorting and filtering must be performed because the data from the images is interlaced and one channel from the ROE will contain fake 
data because there are only three CCDs. This fake data must be filtered out of the system. This software must also keep all of these things
synchronized. 


\subsubsection{Timer \& Uplink Processing}
Timer and Uplink pins let the computer know where it is in the experiment. Software is required to monitor these pins and notify the rest of the system
when something has changed.


\subsubsection{Power System Control}

Power to most of the other subsystems in MOSES is controlled by placing different signals onto control lines connected to the computer. Software for 
controlling these signals is required.

\subsubsection{Process Concurrency}

All of the software for controlling the different systems need to be running concurrently. Furthermore, many tasks within different processes need to
be happening concurrently. This calls for a multithreaded approach. Therefore, the software needs some way of supporting multitasking. It must also ensure
that shared data is threadsafe. Threadsafe basically means that one thread cannot alter data while another thread is in the process of reading that same
data. This ensures that data is not accidentally corrupted.

\subsubsection{System Software Housekeeping}
There are many aspects of the software status that it will be desirable to know during testing and during the experiment. Therefore, some kind of 
housekeeping software is needed to keep track of software status, including such things as current exposure sequence and current position in the 
current exposure sequence. 

\subsubsection{Sufficient Transmission Encoding}
One problem with transmitting a digital signal via telemetry is that the characteristics of the data can affect the probability of the signal being
received correctly. For instance, long strings of 1's can cause the voltage of the signal to wander. This voltage wandering can render the receiving end 
incapable of recovering an accurate clock with which to extract the data. Because of this and similar issues, an encoding of the data is necessary that 
will provide the highest degree of likelihood that the data will be received correctly.

\subsubsection{Autonomy}
Obviously, being part of a rocket payload requires that this software be semi-autonomous. It needs to be able to run the entire experiment, without human
input, based entirely on input from a preprogrammed timer deck. This means that a lot of interprocess communication needs to take place to ensure that
all pieces of software are performing the correct tasks at any point during the experiment.

\subsubsection{Manual Override}

As with anything in real life, there are a thousand possible things that could go wrong during the experiment. Therefore, it is necessary that some form
of manual override be available during the experiment. This manual override should allow someone at the ground station to alter any aspect of the experiment.
The software must provide the ability to control the power subsystem, data acquisition, and ROE, as well as to alter the exposure sequence and to simulate 
Timers and Uplinks. Clearly, this kind of power could be very dangerous so safeguards should be in place to provide such things as error detecting for
the physical link that carries the signal. It could be disastrous if a command sent by the ground station was corrupted during transmission and then 
misunderstood to be a different command by the Flight Computer. All of these issues must be accounted for in the software.







\section{Methods}


\subsection{Hardware}
The majority of the hardware requirements were met with only a handful of commercial components. The following subsections describe these products and
clarify how they fulfilled the requirements.

\subsubsection{PC104+ Architecture}
At this point is important to note that we decided to go with the PC104+ computer architecture. This relatively new form-factor is surprising compact
and power efficient. It also supports a high speed PCI bus for excellent system throughput. PC104+ boards are interesting because unlike most modern 
computers that have slots for add-on cards, PC104+ cards stack on top of each other with the system bus running up through the cards. In this way the
cards form a stack that is the basis for the physical infrastructure.

\subsubsection{EBX Hercules Single Board Computer}
The base system functionality is provided by the EBX Hercules single board computer from Diamond Systems. This card supports a 500Mhz VIA C3 Processor,
which runs surprisingly cool, 128MB SDRAM, four RS232/485 Serial Ports for low speed serial communication, an RJ-45 Ethernet port for umbilical support,
and a Digital I/O pin header that provides digital input/output signal processing.


\subsubsection{PCI FIO Data Acquisition Board}
Data Acquisition is provided by the PCI FIO board from RPA Electronics. This board is interesting in that it provides a programmable FPGA chip which the 
user loads with their own design to acquire the data. The board also contains a FIFO for buffering the data, as well as user software and libraries to 
support reading information from the FIFO on the computer side and loading designs into the FIFO.

The idea of using an FPGA chip was that the card would become a lot more flexible. Users can tailor the design to fit their specific needs. The card also
contains an EEPROM which can be loaded with the design and a jumper can be selected which loads the FPGA from the EEPROM upon startup. That way the design
doesn't need to be loaded into the chip by the user every time the computer is restarted.

The card also uses direct memory access to increase the data throughput. Advertised rates are over 30MBps. 

\subsubsection{SuperFastCom High-speed Serial Output Card}
High-speed serial output is accomplished with the SuperFastCom card from Commtech. This card claims to operate at up to 10Mbps. An added bonus is that
the card can be configured to use an HDLC and NRZI encoding. Without going into to much detail, these encodings are great for ensuring that transmitted
data isn't corrupted. The NRZI encoding prohibits dangerous voltage wandering as well.

 
\subsubsection{E-Disk Solid State Flashdrive}
For solid state storage, we decided to go with the E-Disk Flashdrive from Bitmicro. This drive has a capacity of 1GB and can support throughput of up to
66MBps. This device is also low power and produces very little heat. It requires no air flow for cooling purposes. It is also military grade and can 
withstand a large range of temperatures and acceleration.


\subsection{Software}

With the exception of the operating system and the hardware drivers, all of the software for this project has been written in house. Design of 
the software followed a top-down design and implementation approach. Similar software tasks have been grouped together to form larger applications.
Currently, the software consists of four programs plus the operating system. Each of these is described in further detail in the following subsections.

\subsubsection{Linux Operating System}
We chose to use Linux for our operating system. This seemed to be an obvious choice for a number of reasons. First, Linux is a general purpose O.S. 
that could provide us with all of the power we need. Second, Linux is open source and can be adapted and streamlined to fit our specific needs. One 
important ability this provides us is being able to slim the operating system by keeping only what we need and removing everything else. This
allows us to fit the O.S. onto the flashdisk with 750MB reserved for data.

\subsubsection{Concurrency}
The Pthreads library is a UNIX based system library that provides an easy way to implement multithreading within an application. It also provides
methods of performing some synchronization between threads. For example, mutexes allow locks to be placed around resources. These locks ensure that
two threads can't access the same resource at the same time. The Pthreads library also provides conditional variables which allow threads to wait for
certain conditions to become true. This can be used to synchronize threads that need to perform a group task in a specific order.  

\subsubsection{I/O Server}
The I/O Server provides all of the house keeping information for the computer and also allows for the manual override of any other program. It also
provides a virtual shell, so a user can alter any part of the operating system as well. The protocol for communicating with I/O server enforces a 
stringent error detection policy to keep corrupted commands from damaging the system.

\subsubsection{Power Server}
Controlling the power subsystem is accomplished with the Power Server program. This program utilizes the digital I/O provided by the Hercules EBX board to
communicate with the power subsystem. This program also provides an interface whereby it can be commanded or queried about the power subsystem.

\subsubsection{Virtual Timer and Uplink Deck(VTMUDeck)}
The Virtual Timer and Uplink Deck is a piece of software that monitors the digital I/O pins that are connected to the physical Timer and Uplink Deck. This
program continually polls these pins and uses signals and FIFOs to notify the appropriate applications when a timer or an uplink has been detected.

\subsubsection{Mission Data Acquisition(MDAQ)}
The Mission Data Acquisition(MDAQ) program implements most of the internal data pipeline. It controls the shutter, ROE, PCI FIO, and SFC cards. It 
commands the shutter with the digital I/O on the Hercules board and uses the gettimeofday() system call for high resolution timing. This function 
operates in microseconds, so it is highly likely that we are satisfying the timing requirements of 10ms control and 1ms known accuracy. The MDAQ program
also contains a set of exposure sequences that run when input is received from the VTMUDeck program indicating certain timer events. Command of the
ROE is accomplished with simple commands being sent down the appropriate serial link. Actual data acquisition is provided through the use of a library
provided with the PCI FIO card. After acquisition, the MDAQ program sorts the data into the appropriate images and filters out the image data from the
unused channel. Next, the data is written to disk and then a separate thread of execution continually sends received data to the SuperFastCom card for
transmission. 

The MDAQ program also contains an interface which the I/O server can use to manually control the data acquisition process. Through this interface, 
every aspect of the exposure sequences can be altered. A sequence can even be stopped mid-exposure. Besides controlling the program, the interface
can be used to query for housekeeping data such as current position in sequence or any of a multitude of housekeeping values for the ROE.
 
\section{Results / Analysis}

This project is still in progress. That being the case, many results are still unavailable. 
Understandably, further analysis of much of the project is difficult or impossible at this time. However, the
results that are currently known are very encouraging and will be covered in the remaining subsections. Where possible, an analysis 
of the results has been performed. 

\subsection{Hardware Performance}

In general, the hardware has performed with a high level of quality. We have experienced little hardware difficulties with the equipment that 
we have purchased. For the most part, the system works as it was
intended. 

\subsubsection{Speed}
Speed testing for storage devices can be very difficult due to the nature of the device, so we have been yet unable to acquire any numerical
data about the actual speed of the Bitmicro Flashdisk in practice. What we do know, is that there have been no problems yet with the speed of
the device. The Flashdisk showed no lag under heavy work loads. In fact, the Flashdisk has been one of the most stable pieces of hardware that
we have purchased.

The PCI FIO data acquisition card produced great results when tested for speed. Despite the slightly contrived nature of the benchmark tests, 
the card was still operating at over 30MBps. This well exceeded the 12Mbps required and everyone is satisfied that the speed of the card will
not be an issue.

The overall speed of the computer was also encouraging. Much of this speed was the result of using a commercial CPU. The processor ran fast enough
to provide sufficiently high resolution timing, despite the non-real-time and non-preemptive operating system.
 
\subsubsection{Stability}

The stability of the hardware in general has been very encouraging with the exception of the RS232/485 serial ports. Despite assurance from
vendors of the compatibility between RS485 and RS422, slight differences in hardware characteristics caused seemingly random communication 
errors. Steps are being taken to rectify the situation and to provide truly RS422 serial ports.

\subsubsection{Useability}

Hardware usability was good, in general. However, the PCI FIO card ended up having a very steep learning curve that made it an expensive decision
in terms of man hours. Producing a working design for the FPGA was also incredibly time consuming. This card is incredibly powerful and flexible,
but in the future we would recommend its use only if an FPGA design expert is close at hand or if the power and flexibility is absolutely
necessary.

\subsection{Software Performance}

As is the nature of software, many minor difficulties have been experienced, however these had little effect on this system as a whole and will
not be covered. Instead, only a general summary of the software performance is provided with the exception of several major benefits or consequences.

\subsubsection{Stability}

The operating system has been very stable throughout this project. This is indicative of the general purpose O.S. we chose to run. Linux, in general, 
has been widely used and highly tested. The high-speed underlying hardware also allowed the system to remain highly responsive.	


Concurrency became an issue when it was discovered that Pthreads and system signals have a difficult time 
coexisting within the same code. Fortunately, work-arounds were discovered that bypassed these limitations. Also, it
  was discovered that condition signals that were sent within the system were not queued and could be lost if
  the thread that was waiting on the conditional variable was not yet in the waiting state. These issues required a significant code restructuring.

\subsubsection{Control Accuracy}
Shutter control proved to be very accurate and very repeatable. Accuracy could be controlled to within fractions 
of a millisecond and could be known on an unloaded system to within 4us.

\section{Remaining To-do List}

The most important remaining task is a full system integration in two steps. First, all of the software must be tested for compatibility and can be
guaranteed to work together. Second, the Flight Computer must be integrated with the other subsystems in the payload and must be tested to ensure full
system interoperability.

Another remaining task is a final speed and space optimization scheduled to take place after the software has been thoroughly tested for correct 
functionality. This optimization will ensure that the system remains responsive and can store as much science data as possible.

Of lesser concern is the necessity of acquiring truly RS422 serial ports. This will most likely only require the purchase of serial converters or an
RS 422 add-on board. This is anticipated to be a minor hardware change and software impact is speculated to be minimal.


\section{Conclusions}
In summary, proper operation of the MOSES experiment requires a soundly designed and implemented Flight Computer. The computer is not only responsible for
nearly every stage of the data pipeline, but it must also control and interact with an assortment of other systems within the payload. Hardware has been chosen
to facilitate the physical requirements of the Flight Computer and software has been designed and written to operate the hardware and orchestrate the 
entire experiment. It remains to be seen whether or not this system will work correctly during the experiment, but preliminary testing has been encouraging.
If successful, it is anticipated that this system might be used as the basis for similar systems in the future.


\end{document}

