%\documentclass[11pt]{article}
%\begin{document}
\section{Main Subsystems}
\hrulefill
\\

	\begin{enumerate}
		\item Flight Computer (FC) stack: The FC stack consists of four boards that work together to capture scientific data and control the instrument.
		\begin{enumerate}
			\item Tri-M VDX104+: Main flight computer board. Contains the CPU, SD hard drive, RS-232 ports, and PCI bus. This is the board that executes the flight software.
			\item CTI FreeForm PCI-104 Virtex 5 FPGA: Connected to the VDX104+ through PCI bus. Main function is to capture science data produced by the ROE. Also implemented on the FPGA is all of the output/input GPIO lines required for operation of power subsystems (outputs) and Timers/Uplinks (inputs).
			\item Synclink USB Adapter: Connected to the VDX flight computer. Its job is to send science data using RS-422 protocol to the telemetry section at 10 Mbit/s. \textbf{NOTE: The Synclink USB adapter has been decommisioned for MOSES-III and subsequent missions}.
			\item MOSES Control Interface Board: This board contains the power supply for the flight computer and the ROE SPU and routes all signals and data from the flight computer into the harness.
		\end{enumerate}
		\item MOSES Instrument Electronics: The MOSES instrument relies on several electronic subsystems to achieve it's goal of capturing images of the Sun. These systems operate mostly independently, but rely on the FC stack for control.
		\begin{enumerate}
			\item Charged-Coupled Devices (CCDs): MOSES employs a diffraction grating as its primary optical element to provide data in three spectral orders. The instrument has three CCDs that capture the images from each order independently. Multiple coatings on the grating are able to block out (nearly) all wavelengths besides the target lines of 459 and 465 Angstroms. Each exposure creates three different images with uniquely-shifted overlap of the target lines. These spectral orders are known as Minus, Zero, and Plus; together, they can be used to generate a three-dimensional representation of solar activity.
			\item Read-out Electronics (ROE):  The ROE is the interface between the FPGA and the three CCDs that capture scientific data. Upon receiving a command from the flight computer, it begins readout by clocking the data contained in each pixel of the CCDs and streaming it in serial to the SPU (defined below). The FPGA then buffers and formats converted data into separate channels that can be identified by the flight software. The ROE was not designed specifically for MOSES, but for the Hinode instrument which has four CCDs. As a result, the ROE actually provides a fourth data channel: Noise. Noise is labeled the 0 channel while the Minus, Zero and Plus are respectively labeled 1, 2, and 3.
			\item Serial-to-Parallel Unit (SPU): The SPU converts serial data from the ROE into 16-bit, 32 Mbit/s parallel data stream that will be captured by the FPGA. 
			\item Power board: This board manages the power systems on the instrument. The flight computer applies a 'high' value to whichever power subsystems have been requested to be activated and then strobes a latch to turn it on/off.
			\item High-Speed Telemetry Transmitter: Science data is sent from the FC, through the Synclink USB adapter, Premod filter, and finally to a radio transmitter which sends the data to the ground at 10 Mbit/s.
			\item Timers and Uplinks: Timers and Uplinks are single-ended GPIO lines that are used to provide basic control to the instrument. When triggered, they consist of a 5V rising edge.
			\item Housekeeping Link Protocol (HLP): Consists of two separate radio connections, HKUP and HKDOWN. HKUP operates at 1200 baud and sends uplinks from the ground to the instrument. HKDOWN operates at 9600 baud and sends replies from the instrument to the ground. The HLP link is the main way of controlling the instrument during testing and provides the most control over the instrument.
			\item Shutter: The shutter opens and closes (obviously) to allow light to reach the CCDs for each exposure. The this operation is controlled by two GPIO lines connected to the VDX flight computer. The reason these lines are separate from the rest of the GPIO lines implemented through the FPGA is the developers felt that the FPGA may be busy during the time which the shutter is intended to close.			
		\end{enumerate}
		\item Electronic Ground Station Equipment (EGSE): The EGSE is designed to interpret and display data sent back to the ground. It consists of three computer systems that manage the different types of data provided by the instrument. These systems are located in a server tower with the necessary electronic support as well as power supplies to operate the instrument during testing.
		\begin{enumerate}
			\item EGSE1: The computer labeled EGSE1 is a computer running Microsoft Windows XP that is designed to capture analog housekeeping data produced by the instrument. The software that captures this analog data is written in LabView and provides graphs of temperatures and currents on the instrument over time.
			\item EGSE2: The computer labeled EGSE2 is a computer running Linux Mint 17.1 used for running the EGSE software. This software is a Java program consisting of a Server and Client that work together to exchange HLP packets with the flight computer. EGSE2 is also connected to the flight computer via an ethernet connection while the instrument is on the ground. This allows users to open an SSH session with the flight computer for debugging purposes.
			\item EGSE Laptop: This Ubuntu 14.04 computer runs a program known as receiveTM to capture high-speed telemetry sent by the MOSES instrument. Received data is then piped to an IDL program which displays the images.
		\end{enumerate}
	\end{enumerate}
