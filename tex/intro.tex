\section{Introduction}
\hrulefill
\\
The MOSES instrument is a sounding rocket payload designed to take exposures of the Sun in Extreme Ultraviolet wavelengths. Since the sounding rocket trajectory guarantees only 5 minutes of viable exposure time, the flight software must be able to operate nearly autonomously for the entire flight. Additionally, the flight software must also be able to be controlled from the ground to be able to integrate and test the instrument properly.\par

MOSES first launched in 2006 using flight software developed by Reginald Mead in C++, running on a Hercules EBX flight computer. Unfortunately this old flight computer started to develop electrical problems and could not be replaced. These events necessitated the selection of a new flight computer and development of new flight software. The new configuration was designed very similar to the original configuration in that an FPGA is utilized to capture the experimental data from the cameras, and that data is transferred back to the flight computer using Direct Memory Access. As a result the flight software is similar to the software developed by Reginald, except that it is written in C. \par

While the old flight software certainly performed as expected on the 2006 flight, the software developers for the updated software made many attempts to fix or implement features that were not developed in time for the first flight. These include: reducing the latencies between subsequent exposures, a well tested telemetry module, and good integration between ground station software and flight software. \par

With all that being said, the developer must note that this document is not intended as a Software User's Guide. It is designed to be a resource for the maintainer of the MOSES flight software and to provide a source code level description of the program. While all attempts will be made to make this document as accurate as possible, there will inevitably be errors that creep into this writing. The source code is the main source of information on the operation of the flight software, and an in-depth understanding of the software will only be possible through reading the source.
