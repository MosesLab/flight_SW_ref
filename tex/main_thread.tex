%\documentclass[11pt]{article}
%\begin{document}
\section{Main Thread}
\hrulefill
\\
Program initialization begins with the configuration file \texttt{moses.conf}. Each setting in the file specifies an individual thread or method of communication and is assigned a one (ON) or zero (OFF) for toggling during the debug process. After reading the configuration file, the Main thread starts the \texttt{bash} subprocess and notes its Program ID (PID). This will be used to kill the \texttt{bash} subprocess upon program exit. Then the Main thread proceeds to start each thread specified as active by \texttt{moses.conf}. The order of threads in \texttt{moses.conf} designates their load order; i.e. items listed higher will have higher execution priority. During flight, every option shall be set to ON. 

After each thread has successfully started, the Main thread is paused and simply waits for program exit. To exit the program any one of the threads may send a \texttt{SIGINT} Linux signal to the main thread to inform it of program shutdown. The main thread is waiting for this signal, and once it is raised, it will begin the process of cleaning up threads and exiting the program.