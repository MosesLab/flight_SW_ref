\section{High-Speed Telemetry}
\hrulefill
\\
One of the mission requirements for MOSES II is to ensure the retrieval of meaningful data without relying on the payload's survival. The concern arose when MOSES I landed in an active, unexploded ordnance range (luckily, its detonation was not required and the payload was recovered). Such a backup system would even allow for mission success in the event of in-flight failures. To satisfy the requirement, Synclink boards were implemented to manage the high-speed transfer rates of the provided radio transmitter. One is used on each end (flight $\rightarrow$ ground) to mediate the science data's safe transfer. Custom programs were implemented in the C language to control the Synclinks; "sendTM," which is given its own thread in the flight software, and "receive_TM," a standalone program run by the ground station laptop.

The Telemetry thread is generally initialized with the lowest priority because it has little interference and becomes active last. Its function is to detect objects in the telemetry queue written by the Image Writer thread. When the queue updates and the queue's lock is released, the Telemetry thread accepts the next pointer and begins streaming its data to the on-board Synclink. The modules are separated by the Premod filter, which converts the signal to analog for transfer via the provided RF transmitter (as outlined in section 2e).

Meanwhile on the ground, the Ground Station laptop shall be running the receive\_TM program, which passively monitors the second Synclink's input. When detecting valid packets, the program displays the size of each, processes the incoming signal, and saves the data to disk. Each full transmission (including all three of the valid channels) takes about 10.1 seconds; each index (.xml) file takes approximately 0.5 seconds, but this value obviously increases with the number of entries recorded within (one for each of the completed exposures). Upon success, the data is saved locally before being piped to an external IDL script, which then reconstructs the images and displays them on screen.

